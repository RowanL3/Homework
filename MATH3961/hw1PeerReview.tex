\documentclass[11pt] {article}

\usepackage{amsfonts}
\usepackage{amsmath}
\usepackage{amssymb}
\usepackage{amsthm}
\usepackage{mathtools}
\usepackage{array}
\usepackage{enumitem}

\newcommand{\Z}{\mathbb{Z}}
\newcommand{\T}{\mathcal{T}}
\newcolumntype{C}{>$c<$}
\newcolumntype{L}{>$l<$}

\begin{document}
\noindent Question 1:
\begin{enumerate}
\item
Clear and correct truth table and explanation. However what you proved was the \textit{sequent:} $\vDash (ii) \Rightarrow (i)$ not the \textit{theorem:} $(ii) \vDash (i)$. The subtle distinction between them is that $\vDash$ is not a truth functional connective like $\Rightarrow$, it is a property of the meta-logic so it should not appear in the truth table. Instead you should show that in every configuration $(ii)$ is true $(i)$ must also be true in your explanation. This is probably easiest to do in the contrapositive e.g. the only configuration where $(i)$ is false $(ii)$ must also be false. Same with $(i) \nvDash (ii)$ the argument there being by counter example (there exists some configuration in which (i) is true but (ii) is false).
\item
Largely same as above however what makes this answer far better is you explicitly state, $(ii) \Rightarrow (i)$ is a theorem and $(i)\Rightarrow (ii)$ is not. My only two remaining nit picks are it would be make it a little more clearer how you get from the sequent to the theorem, and (in my eyes at least) proving the Theorem to prove the sequent is a little indirect it would probably be better to just prove the sequent directly.
\item 
Clear, concise and correct.
\end{enumerate}
Question 2a:
\begin{enumerate}
\item
Valid and elegant proof.
\item
It looks like you have the right idea in trying to get $P \land Q$ for modus ponens on $ (P\land Q) \Rightarrow R$ however in line 3 you assume $P \land Q$. meaning that line 6 (where you discharge the assumption) would actually be $(P \land Q) \Rightarrow R $. Which is not what you are trying to prove. Instead the assumption you want is just $P$ and you should be able to build $P \land Q$ from that and your initial assumption.
\item (Out of order) The only mistakes here are that you say you got $Q$ and $(P\land Q) \Rightarrow R$ by assumption (and I suspect that you meant $\land-E$ because wrote that they both depend on line 1) and you forget to remove $P$ from your assumptions on line 7.
\end{enumerate}
Question 2b:
\begin{enumerate}
\item Great proof. My only note is that you could save a step by not discharging the assumption $(P \land R)$ until after the $\lor -E$ step (making Q your conclusion for $\lor -E$).
\item You appear to be on the right track logically however your proof seems to be working under the premise that $B,A\Rightarrow B \vDash A$ when this is not the case. Modus ponens is the idea that $A,A\land B \vDash B$. You seem to have made a similar mistake for conditional proof as well. If begin by assuming $P\land R$ and worked your way to $Q$ instead of the other way around this proof would be totally correct.
\item Valid and elegant proof.
\item Valid and elegant proof.  Gold star for organize your $\lor -E$ argument in cases
\end{enumerate}
Question 2c:
\begin{enumerate}
\item Valid and elegant proof. 
\item It appears you may be confused with the statement of the problem, you are trying to prove $P\lor (Q\land R)$ from $P \lor Q $ and also prove it from $P \lor (Q \land R)$. When what $P\lor Q, P\lor R \vDash P \lor (Q\land R)$ really means is $P\lor (Q\land R)$ is provable from \textit{both} $P\lor Q$ and $P\lor R$. What you have attempted to prove here is $P\lor Q \vDash P \lor (Q \land R)$ \textit{and} $P\lor R \vDash P \lor (Q \land R)$  which are both impossible proofs as it is only from the combination of the two assumptions that you can infer the conclusion (It looks like in your argument you're mistaking $\land-I$ for $\lor-I$ in your attempt to do so). It's often useful to try and reason your way through the sequent before trying to prove it formally.
\item Valid and elegant proof. Looks largely the same as 1, except with a different order of steps. Might have been nice to put the assumption of $P$ nearer the top to highlight the nested  $\lor -E$ structure of the proof but this is a minor stylistic nit pick.
\end{enumerate}
Question 2d:
\begin{enumerate}
\item Correct proof and relatively short. However reductio ad absurdum is not necessary for this proof, a far more direct approach would be to use Modus Tolens from $\sim Q$ to get $\sim P$ in one step. This saves three lines that were spent constructing the contradiction and results in a more succinct proof. 
\item Good proof, the only minor mistake is that the label on line 6 should be modus tollens. Modus ponus is the idea that $A,A \Rightarrow B\vDash B$ when what you actually did was $\sim B, A \Rightarrow B \vDash \sim A$  (modus tollens).
\end{enumerate}
Question 3a:
\begin{enumerate}
\item Good method of investigating which, of the two WWF's was a theorem without having to go into a truth table. 
\end{enumerate}
Question 3b:
\begin{enumerate}
\item Stealer proof, cleaver use of the "anything for a contradiction sequent" in your first $\lor -E$ step and proving the conclusion from $P$ and from $\sim P$ was a fantastic way to structure it. I give it 5 stars and 2 thumbs up! 
\item There are a few flaws in your proof. Firstly if you assume a direct contradiction like $Q\land \sim Q$ everything you infer from it will always in some way be predicated on that contradiction making it useless (because anything follows from a contradiction). Secondly from lines 11 and 12 it looks like you are misunderstanding conditional proof. Conditional proof is the idea that $P \vDash Q$ implies $\ \vDash P \Rightarrow Q$. This is also where you lose track of the assumption of $Q\land \sim Q$. Lastly your proof should always end with everything on the right side of the $\vDash$, with your only assumptions being things on the left side of the $\vDash$.
\end{enumerate}

\end{document}