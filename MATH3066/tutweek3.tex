\documentclass[11pt] {article}

\usepackage{amsmath}
\usepackage{amssymb}
\usepackage{amsthm}
\usepackage{mathtools}

\begin{document}
\begin{flushleft}
Rowan Lochrin\\
MATH3308\\
Week 3 Tutorials \\
23/3/17\\
\end{flushleft}

\begin{enumerate}
\item
\begin{enumerate}
\item This is a fallacy of denying the antecedent
\item This is a valid instance of disjunctive syllogism.
\end{enumerate}
\item
\begin{enumerate}
\item \textit{Not a theorem} if $P$ is false then $(P \Rightarrow \sim P) \Rightarrow \sim P$  is false.
\item \textit{Theorem} $(P \Rightarrow \sim P) $ is only true when $P$ is false so $(P \Rightarrow \sim P) \Rightarrow \sim P$ must be true, as the consequent must be true whenever the antecedent is true.
\item \textit{Theorem} $[(P\Rightarrow Q) \land (R \Rightarrow P)]$ implies $ R \Rightarrow Q $ and by contraposition $ \sim Q \Rightarrow \sim R$. whenever the antecedent is true the consequent is true.
\item \textit{Not a Theorem} We can see if $R,P$ is false and $Q$ is true then  $[(P\Rightarrow Q) \land (R \Rightarrow P)] \Rightarrow (\sim R \Rightarrow \sim Q)$ as the antecedent is true and the consequent is false.
\end{enumerate}
\item
\begin{enumerate}
\item
$$f(n) = n+1$$
\item
$$f(n) = n -1 $$
\item
\[
f(n) =  \begin{cases} 
      \hfill $2n$ \hfill & \text{ if $n$  is not negative} \\
      \hfill $-2n  - 1$ \hfill & \text{ if $n$ is negative} \\
  \end{cases}
  \]
  \item
\[
f(n) =  \begin{cases} 
      \hfill \frac{n}{2} \hfill & \text{ if $n$ is even} \\
      \hfill \frac{-(n - 1)}{2}  \hfill & \text{ if $n$ is odd} \\
  \end{cases}
  \]
\end{enumerate}
\item 


\end{enumerate}


\end{document}