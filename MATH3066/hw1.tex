\documentclass[11pt] {article}

\usepackage{amsfonts}
\usepackage{amsmath}
\usepackage{amssymb}
\usepackage{amsthm}
\usepackage{mathtools}
\usepackage{array}
\usepackage{enumitem}

\newcommand{\Z}{\mathbb{Z}}
\newcommand{\T}{\mathcal{T}}
\newcolumntype{C}{>$c<$}
\newcolumntype{L}{>$l<$}

\begin{document}
\begin{flushleft}
Rowan Lochrin\\
MATH3066\\
Homework 1\\
26/4/17\
\end{flushleft}
\begin{enumerate}
\item
\begin{enumerate}
\item
\[
\begin{array}{C|C|C|C|C}
$P$ & $Q$ & $R$ & $(P\land Q)$ & $(P\land Q)\Rightarrow R$\\ 
\hline
T & T & T & T & T\\
T & T & F & T & F\\
T & F & T & F & T\\
T & F & F & F & T\\
F & T & T & F & T\\
F & T & F & F & T\\
F & F & T & F & T\\
F & F & F & F & T
\end{array}
\]
\item
\[
\begin{array}{C|C|C|C|C|C}
$P$ & $Q$ & $R$ & $\sim P$ & $(\sim P\land R)$ & $Q \Rightarrow (\sim P\land R) $\\ 
\hline
T & T & T & F & F & F \\
T & T & F & F & F & F\\
T & F & T & F & F & T\\
T & F & F & F & F & T\\
F & T & T & T & T & T\\
F & T & F & T & F & F\\
F & F & T & T & T & T\\
F & F & F & T & F & T
\end{array}
\]
\item
 $ Q \Rightarrow (\sim P \land R) \vDash (P\land Q) \Rightarrow R $:\newline
Because in order for $V(( P \land Q) \Rightarrow R) = F$, $V(P) = V(Q) = T$ and $V(R) = F$ meaning $ V(Q \Rightarrow (\sim P \land R)) = F$.\newline
\item$ \ (P\land Q) \Rightarrow R \nvDash Q \Rightarrow (\sim P \land R) $:\newline
as if $ V(P) = V(Q)  = V(R) = T $ then $V(( P \land Q) \Rightarrow R) = T$ and  $ V(Q \Rightarrow (\sim P \land R)) =F $


\end{enumerate}
\item
\begin{enumerate}
\item
\[
\begin{array}{LLLL}
1 & (1) & $ ((P \land Q) \Rightarrow R) \land Q $ &  $RA$ \\
1 & (2)&$ Q $ & $ 1,\land -E $\\
3 &(3) &$ P $ & $RA$\\
1,3 &(4)&$ (P\land Q) $& $2,3\land -I $\\
1 &(5)&$(P \land Q) \Rightarrow R $ & $ 1,\land -E $\\
1,3 &(6) &$ R $& $4,5 MP $\\
1 &(7)&$ P\Rightarrow R $& $6 \Rightarrow -I $\\
\end{array}
\]

\item
\[
\begin{array}{LLLL}
1 & (1)  & $(P \Rightarrow Q) \lor (R \Rightarrow Q) $ &  $RA$ \\
2   & (2) &$(P \Rightarrow Q)$ & $RA$ \\
3   & (3) &$(R \Rightarrow Q)$ & $RA$\\
4 & (4)  & $ (P\land R) $ & $RA$\\
4 & (5)  & $ P $ & $2 \land-E$\\
4 & (6)  & $ R $ & $2 \land-E$\\
2,4  &(7)  & $ Q $& $2,5MP $\\
3,4  &(8)  & $ Q $& $3,6MP $\\
1,4 &(9)  & $ Q $ & $1,2,3,7,8\lor-E$\\
1 &(10)  & $ (P\land R) \Rightarrow Q $ & $1,4CP$\\
\end{array}
\]

\item
\[
\begin{array}{LLLL}
1 & (1)  & $(P\lor Q) $ &  $RA$ \\
2 & (2)  & $ (P \lor R) $ & $RA$\\
2 & (3)  & $ P $ & $2 \land-E$\\
2 & (4)  & $ R $ & $2 \land-E$\\
1,2 &(5)  & $ Q $& $1,3,4 \lor -E $\\
1 &(6)  & $ (P \land R) \Rightarrow Q$ & $2,5 CP$\\
\end{array}
\]
\item
\[
\begin{array}{LLLL}
1 & (1)  & $(P\Rightarrow Q) \land (R\Rightarrow \sim Q) $ &  $RA$ \\
2 & (2)  & $ R $ & $RA$\\
1 & (3)  & $ (P \Rightarrow Q) $ & $1 \land-E$\\
1 & (4)  & $ (R \Rightarrow \sim Q) $ & $1 \land-E$\\
1,2 & (5)  & $ \sim Q $& $2,4 MP $\\
1,2  &(6)  & $ \sim P $ & $3,5 MT$\\
1, & (7) & $ R \Rightarrow \sim P $ & $ 2,6 CP $\\
\end{array}
\]

\end{enumerate}

\item 
\begin{enumerate}
\item

\[
\begin{array}{C|C|C|C|C|C}
$P$ & $Q$ & $R$ & \begin{tiny} $ (P \Rightarrow (\sim Q \land R))$  \end{tiny} & \begin{tiny} $ (Q \Rightarrow ( \sim P \land R)) $ \end{tiny} &  \begin{tiny} $ (P \Rightarrow (\sim Q \land R)) \Rightarrow ( Q \Rightarrow ( \sim P \land R))$ \end{tiny}   \\ 
\hline
T & T & T & F & F & T \\
T & T & F & F & F & T\\
T & F & T & T & T & T\\
T & F & F & F & T & T\\
F & T & T & T & T & T\\
F & T & F & T & F & F\\
F & F & T & T & T & T\\
F & F & F & T & T & T
\end{array}
\]
We can see from the truth table above that $$ (P \Rightarrow (\sim Q \land R)) \Rightarrow ( Q \Rightarrow ( \sim P \land R)) =  F $$ when:  $$ P = F, \space Q = T, \space R = F$$
We know this is the only counter-model as the truth table is exhaustive.
\item
\[
\begin{array}{LLLL}
1 & (1)  & $ P\Rightarrow (\sim Q \lor R) $ &  $RA$ \\
2 & (2)  & $ Q $ & $RA$\\
3 & (3)  & $ \sim (\sim P \lor R) $ & $ RA $ (For RAA)\\
2 & (4)  & $ (P \land \sim R) $ & $3 SI(S)$ (De Morgan's Law)\\
3 & (5)  & $ P $ & $ 5 \land-E$ \\
3 & (6)  & $ \sim R $ & $ 5 \land-E$ \\
1,3 & (7)  & $ (\sim Q \lor R) $ & $ 1,7MP$\\
2,3 & (8)  & $ (Q \land \sim R) $ & $ 2,8\land-I$ \\
2,3 & (9)  & $ \sim (\sim Q \lor R)  $ & 8 $ SI(S)$ (De Morgan's Law) \\
1,2,3 & (10) & $(\sim Q \lor R) \land \sim (\sim Q \lor R) $ & $ \land -I $\\
1,2,3 & (11) & $ \sim Q \lor R $ & $ \land -E $ \\
1,2,3 & (11) & $ \sim (Q \sim Q \lor R) $ & $ \land -E$  \\
1,2 & (12) & $ \sim \sim (\sim P \lor R) $ & $3,11,12 RRA $ \\
1,2 & (13) & $ (\sim P \lor R) $ & $ 12DN $ \\
1 &  (14) & $ Q \Rightarrow (\sim P \lor R)$ & $ 2,13CP$\\
 & (15)  & $ (P \Rightarrow (\sim Q \lor R)) \Rightarrow (Q \Rightarrow (\sim P \lor R)) $ & $ 1,14CP$ \\

\end{array}
\]
\end{enumerate}
\item
\textit{Conjecture:} for any WFF W:$$\#W = 3c(W) + v(W) $$
\begin{proof}
\textit{Base Case:} \newline Assume: A is a WWF consisting of a single propositional variable then $\#A = 1$, $v(A) = 1$ and  $c(A) = 0$ so $$\#A = 3c(A) + v(A)$$. \newline
\textit{Inductive Step:} \newline Assume: N,M are WFF's such that $\#N = 3c(N) + v(N)$ and $\#M = 3c(M) + v(M)$:
\begin{align*}
3c((\sim N)) + v((\sim N)) & =  \#N + 3 =  \#(\sim N)\\
3c((M \lor N)) + v((M \lor N)) & = \#N + \#M + 3 =  \#(M \lor N)\\
3c((M \land N)) + v((M \land N)) & = \#N + \#M + 3 =  \#(M \land N)\\
3c((M \Rightarrow N)) + v((M \Rightarrow N)) & = \#N + \#M + 3 =  \#(M \Rightarrow N)\\
3c((M \Leftrightarrow N)) + v((M \Leftrightarrow N)) & = \#N + \#M + 3 =  \#(M \Leftrightarrow N)
\end{align*} 
Because every WFF is either a single propositional variable, a negated WFF, or two WFF's conjoined by a logical connective mathematical induction proves our conjecture  
 
\end{proof}
\item
\begin{enumerate}
\item
\[
\begin{array}{C|C|C}
$P$ & $Q$ & $(P \Rightarrow Q) $\\
\hline
T & T & T \\
T & F & F \\
F & T & T \\
F & F & T \\
\end{array}
\]
We can see from the truth table that if $V(Q) = F $ then $V(P) = F \text{ implies } V(P\Rightarrow Q) = T$ and $ V(P) = T \text{ implies } V(P \Rightarrow Q) = F$ so  $V(P \Rightarrow Q) = V(\sim P)$ \newline
similarly  if $V(P) = T $ then $V(Q) = T \text{ implies } V(P\Rightarrow Q) = T$ and $ V(Q) = F \text{ implies } V(P \Rightarrow Q) = F$ so  $V(P \Rightarrow Q) = V(Q)$
\item
if $V(X) = T$ then:
\[
\begin{array}{L|L}
$V((X \Rightarrow \sim Y) \Rightarrow \sim (Z \Rightarrow X)) =  $ & \\
$V((X \Rightarrow \sim Y) \Rightarrow \sim T) =  $ & By the truth table for $\Rightarrow $\\
$V((X \Rightarrow \sim Y) \Rightarrow F) = $ \\
$V(V(\sim Y) \Rightarrow F) = $  & by part a \\
$V(\sim\sim Y) = $ & by part a \\ 
$V(Y)$
\end{array} 
\]
if $V(X) = F $ then:
\[
\begin{array}{L|L}
$V((X \Rightarrow \sim Y) \Rightarrow \sim (Z \Rightarrow X)) = $  \\
$V((X \Rightarrow \sim Y) \Rightarrow \sim V(\sim Z))  = $&   by part a\\
$V((X \Rightarrow \sim Y) \Rightarrow V(Z)) = $ & \\
$V(T \Rightarrow V(Z))=  $ & By the truth table for $ \Rightarrow $ \\
$V(Z)  $ & by part a
\end{array}
\]
so:
\[
V(W_{X,Y,Z})=  \begin{cases} 
      \hfill V(Y) \hfill & \text{  if $ V(X) = T $} \\
      \hfill V(Z) \hfill & \text{  if $ V(X) = F $} \\
  \end{cases}
 \]
 
\item 
\[
\T_{W_{P_1,Y,Z}}=  \begin{cases} 
      \hfill \T_Y \hfill & \text{  if $ V(P_1) = T $} \\
      \hfill \T_Z \hfill & \text{  if $ V(P_1) = F $} \\
  \end{cases}
 \]

\item 
\begin{proof}
Note that: 
\begin{align*}
V(\phi \land \psi) & = V(\sim (\phi \Rightarrow \sim \psi)) \\
V(\phi \lor \psi) & = V(\sim \phi \Rightarrow \psi)\\
V(\phi \Leftrightarrow \psi) & = V((\phi \Rightarrow \psi) \land (\phi \Rightarrow \psi))
\end{align*} 
For any WFF's $\phi$ and $\psi$ . Therefore by induction all WWF's can be expressed in terms of $\sim$ and $\Rightarrow$ 
\end{proof}
\end{enumerate}
\item
In $\Z_{11}$:
\begin{align*}
3x  -  y  =  & 2\\
7x  +  2y  = & 0
\end{align*}
$$\Rightarrow 13x = 4$$
$$\Rightarrow 2x = 4$$
$$\Rightarrow x = 2$$
$$\Rightarrow y = 4 $$
However in $Z_{13}$:
$$13x = 4$$
$$\Rightarrow 0x = 4$$
Which clearly has no solution so, the system does not have a solution in $ \Z_{13} $
\item
\textit{Conjecture:} The only integer solution to $ x^2+5y^2 = 3z^2 $ is $(x,y,z) = (0,0,0)$
\begin{proof}
Suppose there exists an $ (x,y,z) \neq (0,0,0) $ that solves $ x^2+5y^2 =3z^2$ \newline
If $gcd(x,y,z) \neq 1$. then $(x,y,z)$ can be expressed as $(gx',gy',gz')$ where $ g = gcd(x,y,z)$
so: $$ (gx')^2 + 5(gy')^2 =3(gz')^2 \Rightarrow  g^2(x'^2 + 5y'^2) =3g^2z'^2 \Rightarrow x' + 5y' = 3z' $$
Meaning there exists a solution $(x,y,z) $ such that $gcd(x,y,z) = 1.$\newline
Also note that: $$ x^2+5y^2 = 3z^2 \mod 4 \Rightarrow x^2+y^2 = -z^2 \mod 4 \Rightarrow x^2+y^2+z^2 = 0 \mod 4 $$
Because in modulo  4: $$ x,y,z \in \{0,1,2,3\} \Rightarrow x^2,y^2,z^2 \in \{0,1\} $$  
Meaning that:  $$x^2+y^2+z^2 = 0 \mod 4 \Rightarrow x^2=y^2=z^2 = 0 \mod 4$$
so $x=y=z=0 \mod 2$ implying for any solution $(x,y,z)$, $gcd(x,y,z) \geq 2$. 
Which is a contradiction.    
\end{proof}
\end{enumerate}
 \end{document}