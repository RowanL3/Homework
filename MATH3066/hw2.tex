\documentclass[11pt] {article}
\usepackage[a4paper, margin = 1in,footskip =0.25in]{geometry}
\usepackage{amsfonts}
\usepackage{amsmath}
\usepackage{amssymb}
\usepackage{amsthm}
\usepackage{mathtools}
\usepackage{array}
\usepackage{enumitem}
\newcommand{\C}{\mathbb{C}}
\newcommand{\R}{\mathbb{R}}
\newcommand{\Z}{\mathbb{Z}}
\newcommand{\T}{\mathcal{T}}
\newcolumntype{C}{>$c<$}
\newcolumntype{L}{>$l<$}

\begin{document}
\begin{flushleft}
Rowan Lochrin\\
MATH3066\\
Homework 2\\
31/5/17\\
\end{flushleft}
\begin{enumerate}
	\item 
	\begin{enumerate}
		\item  $(\exists x)(F(x) \land (\forall y)G(x,y)) \vdash (\forall y)(\exists x)(F(x) \land G(x,y)) $
		\[
		\begin{tabular} {L|L|L|LL}
			\hline
			1 & (1) & (\exists x)(F(x) \land (\forall y)G(x,y)) & A &\\
			2 & (2) & F(a) \land (\forall y) G(a,y) & A&\\
			3 & (2) & F(a) & 2 &\land-E\\
			4 & (2) & (\forall y)G(b,y) & 2&\land-E \\
			5 & (2) & G(a,b) & 4&\forall-E \\
			6 & (2) & F(a) \land G(a,b) & 3,5 & \land-I \\
			7 & (2) & \exists x(F(x) \land G(x,b)) & 6 & \exists-I   \\
			8 & (1) & (\exists x)(F(x) \land G(x,b)) & 1,2,6 & \exists-E \\
			9 & (1) & (\forall y)(\exists x)(F(x) \land G(x,y)) & 8 & \forall-I \\
		\end{tabular}
		\]
		\item $(\forall a)((F(x) \lor G(x)) \Rightarrow H(x))  \vdash  (\exists x) \sim F(x) $ 
		\[
		\begin{tabular} {L|L|L|LL}
			\hline
			1 & (1) & (\forall a)((F(x) \lor G(x)) \Rightarrow H(x)) & A & \\
			2 & (2) & (\exists x)\sim H(x) & A &  \\
			3 & (3) & \sim H(a) & A &  \\
			4 & (1) & (F(a) \lor G(a)) \Rightarrow H(a) & 1 & \exists-E \\
			5 & (1,3) & \sim (F(a) \lor G(a)) & 3,4 & MT \\
			6 & (6) & F(a) & A &  \\
			7 & (6) & F(a) \lor G(a) & 6 & \lor-I \\
			8 & (1,3,6) & \sim (F(a) \lor G(a)) \land (F(a) \lor G(a)) & 5,7 & \land-I \\
			9 & (1,3) & \sim F(a) & 1,3,6,8 & AA \\
			10 & (1,3) & (\exists x) \sim F(x) & 9 & \exists-I \\
			11 & (1,2) & (\exists x) \sim F(x) & 2,3,10 & \exists-E \\
 		\end{tabular}
		\]
		\item $(\forall a) (\forall y)(H(y,y) \Rightarrow \sim H(y,y))\vdash  (\forall a) \sim H(x,x) $
		\[
		\begin{tabular}{L|L|L|LL}
			\hline
			1 & (1) & (\forall a) (\forall y)(H(y,y) \Rightarrow \sim H(y,y)) & A &  \\
			2 & (1) & (\forall y) (H(y,y) \Rightarrow \sim H(y,y) & 1 & \forall - E \\
			3 & (1) & H(a,a) \Rightarrow \sim H(a,a) & 2 & \forall - E \\
			4 & (4) & H(a,a) & A &  \\
			5 & (1,4) & \sim H(a,a) & 3,4 & MP \\
			6 & (1,4) & H(a,a) \land \sim H(a,a) & 5,6 & \land - I \\
			8 & (1) & \sim H(a,a) & 1,4,6 & AA \\
			9 & (1) & (\forall a) \sim H(x,x) & 8 & \forall - I \\
		\end{tabular}
		\]
	\end{enumerate}
	\item 
	\begin{enumerate}
		\item
		\begin{enumerate}
			\item The theorem that this is attempting to prove is correct however: $$\sim \forall a \sim G(x) \vdash \sim \sim G(a)$$ is not a valid instance of universal instantiation because $\forall x$ does not appear in the front of the WFF. 
			\item The problem here is in the existential generalization on line 9. The author seeks to replace the assumption of $G(a)$ with $(\exists x)G(x)$. This is not a valid instance of existential generalization as $a$ appears in one of the assumptions for line 9 (line 2: $\sim G(a)$).
		\end{enumerate}
	\item Consider $U = \{a,b\}$ let:
		$$F = \{a\}$$
			$$G = \{b\}$$
		We can see that $(\exists x)F(x) = T$ and $(\exists x)G(x) = T$, so the antecedent of or sequent is true. However $\sim G(a) \Rightarrow \sim F(a) = F$ so the consequent must be false.
	\end{enumerate}
	\item 
	\begin{enumerate}
		\item Note that in a model with only one element for any WFF involving letters a,b, a = b. Meaning:
			\begin{align*}
				W_1 & \rightarrow (\exists x)H(x,x) \\
				W_2 & \rightarrow  (\forall a)(H(x,x) \Rightarrow \sim H(x,x)) \\
			\end{align*}
		So if our model has only one element, a, by $W_1$ we know $H(a,a)$ but $H(a,a)$, implies $\sim H(a,a)$ by $W_2$ so we have a contradiction.
	\item If $a = (x_1, x_2) \in \mathcal{U} \times \mathcal{U}$ if $x \notin K$ then $x \in H$ by $W_3$ so $x \in H \cup K$. If $x \in H\cup K$ then $x \in \mathcal{U} \times \mathcal{U}$ by definition. So $\mathcal{U} \times \mathcal{U} = H\cup K$. $H$ and $K$ are disjoint as any member of $K$ cannot be a member of H again by $W_3$. 
	\item $W_2$ implies that no elements of the diagonal relation are in $H$ (see part a).$W_3$ implies that every element not in $H$ must be in $K$.
	\item $\mathcal{U}$ cannot have 0 elements by definition, it can't have 1 element by part a. Suppose $\mathcal{U}$ has 2 elements - $a,b$ - than by $W_1$ without loss of generality $H(a,b)$, so by $W_3$, $\sim K(a,b)$. Meaning that by the second disjunct of $W_4$ either $H(b,a) \land H(a,a)$ or $H(b,b) \land H(b,y)$. However because we have $H(a,b)$ $W_2$ gives us $\sim H(b,a)$  so neither of these can be true, giving us a contradiction.
	\item Consider $ U = \{a,b,c\} $:
		\begin{align*}
			H &= \{(a,b),(b,c),(c,a)\} \\
			K &= (\mathcal{U}\times \mathcal{U}) \setminus H\\
		\end{align*}
			This is a valid model. There must always be $3$ elements of $H$ when there are $3$ elements in $\mathcal{U}$. $W_1$ tells us that there is at least one element of $H$. $W_3$ and $W_4$ tell us that if $\exists (a,b) \in H$ then there must also $\exists (b,c),(c,a) \in H$ so the number of elements in $H$ must be a multiple for 3. If there were 6, or 9, elements in $H$ then by $W_2$ there would have to be 12 or 18 elements in $\mathcal{U} \times \mathcal{U}$ which contradicts our assumption that there are 3 elements in $\mathcal{U}$, so there must be 3 elements in $H$.
	\end{enumerate}
	\item
	\begin{enumerate}
		\item $x = \frac{2}{3} $ in $\Z_{13}$ where $3x = 2 \mod 13$ meaning:
			$$x = 2*3^{-1} \mod 13 = 2*9 \mod 13 = 5$$
		\item $x = \frac{2}{3} $ in $\Z_{12}$
			$$ 3x = 2 \mod 12$$
			Because $gcd(3,12) = 3 > 2$ this equation has no solution.
		\item $x = \frac{6}{9}$ in $\Z_{12}$
			$$ 9x = 6 \mod 12 \rightarrow 3x = 2 \mod 4 \rightarrow x = 2 \mod 4$$
			Meaning that 2, 6 and 10 all solve this equation.
		\item $ x = \frac{6}{9}$ in $\Z_{16}$
			$$ x = 6*9^{-1} \mod 16 \rightarrow x = 6 * 9 \mod 16 \rightarrow x = 6 \mod 16$$
	\end{enumerate}
	\begin{enumerate}
		\item Assume $p(x)$ is reducible then for some $a,b$:
			$$ x^2 + 1 = (x + a)(x + b) = x^2 + (a + b)x + ab $$
	However there are no two elements $a,b \in \Z_3$ such that $a + b = 0$ and $ab = 1$.
\item $x+1$ is a primitive root as:
	$$ \{(x+1)^n: n \in [1,2...9]\} = R$$
	That is to say $x+1$ spans $R$
      $x$ is not a primitive root as $x^5 = x$ so it does not span $R$.
		\item in $R$ $(2x+1)^2 = x$ and $(x+2)^2 = x$ so these are both square roots of x.
		\item The only solutions of this equation are the square roots of x ($2x+1$ and $x+2$) as the equation can only be factored when $\alpha^2 = x$.
	\end{enumerate}
\item 
	\begin{enumerate}
		\item Consider the homomorphism $\phi:\R[x] \rightarrow \C$ defined by $\phi(P) =P(\sqrt{k}i)$. For all elements $(a + bi) \in \C$ There exists an element ($\frac{b}{\sqrt{k}}x + a) \in \R[x]$ such that $\phi(\frac{b}{\sqrt{k}}x + a) = a + bi$ so $\phi$ is onto meaning $\text{im}\phi = \C$. If $\phi(P) = 0$ then $\sqrt{k}i$ is a root of $P$ so $P$ must be divisible $x^2 +k$ meaning $\text{ker} \phi = (x^2+k)R[x]$. So by the first isomorphism theory $\R[x]/(x^2+k)\R[x] = \C$. \\
			Consider the homomorphism $\phi:\R[x] \rightarrow \R \oplus \R$ defined by $\phi(P) = (P(\sqrt{k}),-P(\sqrt{k}))$. This homomorphism is onto and its kernel $(x^2 -k)\R[x]$. So by the first isomorphism theory $\R[x]/(x^2-k)R[x] = \R \oplus \R$.
		\item Because $\C$ is not isomorphic to $\R \oplus \R$, by part a $S_1$ cannot be isomorphic to $S_2$. Assume $S_3$ is isomorphic to $S_2$ by the first isomorphism theory there exists an isomorphism $\phi$ such that $\text{ker}\phi = (x^2)\R[x]$ and $\text{in}\phi = S_2$. Define $\phi'$ to be $\phi'(x) = \phi^{-1}(x) + 1$ meaning $\text{ker}\phi'= (x^2)\R[x]$ and $\text{in}\phi' = S_1$. So again by the first isomorphism theory $S_1$ must also be isomorphic to $S_3$. Because $3_2$ and $S_3$ are not isomorphic themselves $S_3$ cannot be isomorphic to both of them so $S_3$ must not be isomorphic to $S_2$. $S_3$ cannot be isomorphic to $S_1$ by similar reasoning. 
	\end{enumerate}
\end{enumerate}
\end{document}
