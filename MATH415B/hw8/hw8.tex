\documentclass[11pt]{article}

\usepackage{scrextend}
\usepackage[a4paper, margin = 1.25in,footskip =0.25in]{geometry}
\usepackage{enumitem}
\usepackage{amsmath}
\usepackage{amssymb}
\usepackage{amsthm}
\usepackage{mathtools}
\usepackage{amsmath}
\usepackage{graphicx}
\usepackage{amstext} 
\usepackage{array}


\newcommand{\C}{\mathbb{C}}
\newcommand{\Q}{\mathbb{Q}}
\newcommand{\R}{\mathbb{R}}
\newcommand{\Z}{\mathbb{Z}}

\graphicspath{ {images/} }


\DeclarePairedDelimiter\ceil{\lceil}{\rceil}
\DeclarePairedDelimiter\floor{\lfloor}{\rfloor}

\newcolumntype{L}{>{$}l<{$}} % math-mode version of "l" column type
\begin{document}


\begin{flushleft}
Rowan Lochrin \\
MATH415B - Klaus Lux \\
3/26/1 \\
Homework 8
\end{flushleft}

\section{Gallian}
\begin{description}
	\subsection{Chapter 20}
	\item[27] Prove or disprove that $Q(\sqrt{3})$ and $Q(\sqrt{-3})$ are ring
		isomorphic. \\
	$Q(\sqrt{3})$ is not isomorphic to  $Q(\sqrt{-3})$ 
		\begin{proof}
		Let $\phi:Q(\sqrt{-3}) \rightarrow Q(\sqrt{3}) $ be an
		isomorphic mapping.
		\begin{align*}
			\phi(3+\sqrt{-3}^2) & = \phi(0)\\
			& = 0\\
			\phi(3+\sqrt{-3}^2) & = \phi(3) + \phi(\sqrt{-3}^2)\\
				 & = \phi(1) + \phi(1) + \phi(1) +
				\phi(\sqrt{-3})\phi(\sqrt{-3}) \\
			 & = 1 + 1 + 1 + \phi(\sqrt{-3})^2 \\
			& = 3 + \phi(\sqrt{-3})^2 = 0 
		\end{align*}
		Because $\phi$ is onto there exists $a \in Q(\sqrt{3})$ such
		that  $a^2 = -3 $ and $Q(\sqrt{3}) \subseteq
		\R$ so  $a \in \R$.
		\end{proof}
	\item[28] For any prime $p$, find a field of characteristic $p$ that is
		not perfect. \\
		Consider the Field $Z_p(\sqrt[p]{2})$, this field clearly has
		characteristic $p$ and 
		$$ Z_p(\sqrt[p]{2}) = \{a+b\sqrt[p]{2}|a,b \in Z_p\}$$
		Also
		$$(a + b\sqrt[p]{2})^p = \sum_{k=0}^p\binom{p}{k} a^{n-k}b\sqrt[p]{2}^k$$
		Because $p$ is prime, $p$ divides $\binom{p}{k}$ for all
		values of $k$ except $k = 0$ and $k = p$. All but the first and
		last terms of the expansion will be 0 so for any element of
		$Z_p(\sqrt[p]{2})$.
		$$(a + b\sqrt[p]{2})^p = a^p + (b\sqrt[p]{2})^p = (a^p +2b^p) \in
		Z_p$$
		Implying $ [Z_p(\sqrt[p]{2})]^p \neq  Z_p(\sqrt[p]{2})$.
	\item[30] Show that $x^4 + x + 1$ does not have multiple zero's in any
		extension field of $Z_2$.\\
		\begin{align*}
			f(x) &= x^4 + x + 1 \\
			f'(x) &= 4x^3 + 1 = 1
		\end{align*}
		$$\gcd(f(x), f'(x)) = 1$$
		So $f(x), f'(x)$ do not have a common factor of positive
		degree. The result follows by Cirterion for Multiple Zeros.
	\item[33] Let $F$ be a field of characteristic $p \neq 0$. Show that the
		polynomial ring $f(x) = x^{p^n}-x$ over $F$ has distinct
		zeros.
		\begin{align*}
			f(x) &= x^{p^n}-x\\
			f'(x) &=p^nx^{p^n-1}-1 = 0^nx^{p^n-1} = -1
		\end{align*}
		$$\gcd(f(x), f'(x)) = 1$$ so $f(x)$ cannot have multiple zeros
		in any extension field of $F$ since $F$ was an arbitrary field
		we know that it cannot have multiple zeros in any field. And
		clearly $1$ is a zero in any field.
	\item[34] Find the splitting field for $f(x) = (x^2+x+2)(x^2+2x+2)$ over
		$ F = Z_{3p} $ and write $f(x)$ as a product of linear factors.
		$F(\sqrt{i})$ is a splitting field of $f(x)$
		$$f(x) = (x^2+x+2)(x^2+2x+2) = x^4 + 3x^3 + 6x^2 + 6x + 4 = x^4 +
		1 $$
		And in $F(\sqrt i)$
		$$ x^4 + 1 = (x+\sqrt i)(x-\sqrt i)(x + i\sqrt i)(x - i \sqrt i) $$
	\subsection{Chapter 21}
	\item[2] Let $E$ be the algebraic closure of $F$. Show that every
		polynomial in $F[x]$ splits in $E$.\\
		Let $f(x) \in F[x]$ and let $a_0,a_1,...,a_n$ be roots of
		$f(x)$ then $a_0,a_1,...,a_n \in E$ so in $E$, 
		$$f(x) = (x-a_0)(x-a_1)....(x - a_n)$$
	\item[8] Find the degree of a basis for $Q(\sqrt3 + \sqrt5)$ over
		$Q(\sqrt{15})$. Find the degree and a basis for $Q(\sqrt2,
		\sqrt[3]{2},\sqrt[4]{2})$ over $Q$.\\
		The set $\{1,\sqrt{3}\}$ is a basis for $Q(\sqrt3+\sqrt5)$ over
		$Q(\sqrt{15})$. $\sqrt5 = \frac{1,\sqrt{15}}{\sqrt{3}}$ so $\sqrt{3},
		\sqrt{5} \in Q(\sqrt{15})(\sqrt{3})$ hence any linear
		combination of the two is also in $Q(\sqrt{15})(\sqrt3)$. They
		are also linearly independant as $a+b\sqrt3 = 0$ has no
		solutions for rational $a,b \in Q(\sqrt{15})$
		$\{1,\sqrt[4]{2},\sqrt[3]{2}\}$ is a basis for $Q(\sqrt 2,
		\sqrt[3]{2}, \sqrt[4]{2})$, as
			$$ \sqrt 2  = \sqrt[4]{2}^2 $$
		If the basis is linearly depended hence there exists some
		$a$,$b$, and $c$ in $Q$ such that 
		$$a + b\sqrt[4]2 + c \sqrt[3]2 = 0$$
		which clearly has no solutions.
				
	\item[10] Let $a$ be a complex number that is algebraic over $Q$ show
		that $\sqrt a$ is algebraic over $Q$. Why does this prove that
		$\sqrt[2]{a}$ is algebraic over $Q$?\\
		If $a$ is a algebraic over $Q$ there exists some $f(x) \in Q[x]$
		such that $f(a) = 0$. Let 
		$$f(x) = c_1a^1 +c_2a^2 +... +c_na^n $$
		We can see that
	\item[14] Find the minimal polynomial for $ \sqrt{-3} + \sqrt 2 $ over
		$Q$.\\
	\end{description}
\section{GAP}
	\begin{description}
	\item[22.1]
	\item[22.2]
	\item[22.3]
	\end{description}
\end{document}
