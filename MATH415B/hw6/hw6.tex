\documentclass[11pt]{article}

\usepackage{scrextend}
\usepackage[a4paper, margin = 1in,footskip =0.25in]{geometry}
\usepackage{enumitem}
\usepackage{amsmath}
\usepackage{amssymb}
\usepackage{amsthm}
\usepackage{mathtools}
\usepackage{amsmath}
\usepackage{graphicx}

\newcommand{\C}{\mathbb{C}}
\newcommand{\Q}{\mathbb{Q}}
\newcommand{\R}{\mathbb{R}}
\newcommand{\Z}{\mathbb{Z}}

\graphicspath{ {images/} }

\DeclarePairedDelimiter\ceil{\lceil}{\rceil}
\DeclarePairedDelimiter\floor{\lfloor}{\rfloor}

\begin{document}

\begin{flushleft}
	Rowan Lochrin \\
	MATH415B - Klaus Lux \\
	2/26/18 \\
	Homework 6
\end{flushleft}
\begin{description}
	\section{Gallian}
	\item[11] Trace the argument in example $7$ to find $q$ and $r$ in
		$Z[i]$ such that $3-4i = (2+5i)q+r$ and $d(r) < d(2+5i)$\\
		Note that in $Q[i]$, 
		\begin{align*}
			(3-4i)(2+5i)^{-1} &= \frac{1}{29}(-14-23i)\\
			& = (-i) + (\frac{-14}{29} + \frac{6}{29}i)
		\end{align*}
		So
		\begin{align*}
		(3-4i)  &= (-i)(2+5i)+ (\frac{-14}{29} + \frac{6}{29}i)(2+5i)\\
				 &= (-i)(2+5i) + (-2-2i)
		\end{align*}
		So $q = -i, r= -2-2i$, verify 
		$$d(r) = 2^2 + 2^2 = 8 < d(2+5i) = 2^2 + 5^2 = 29$$
	\item[20]
		Prove that $Z[\sqrt{-3}]$ is not a \textbf{PID}.
		\begin{proof}
			Because \textbf{PID} $\rightarrow$ \textbf{UFD} it will
			suffice to show $Z[\sqrt{-3}]$ is not a \textbf{UFD}
			Consider $4 = 2^2 = (1+\sqrt{-3})(1-\sqrt{-3})$ \\
			We can see that $d(2) = 2^2 + 0 = 4$ so if $x$ is a
			non-unit factor of $2, d(x) = 2 $ meaning  for some
			$a,b$, $a^2 + 3b^2 =2$ which has no solutions.
			So $2$ is irreducible. 
			$d(1+\sqrt{-3}) = 1^2 + 3(1^2) = 4 $ so again if $x$
			is a factor of $(1+\sqrt{-3}, d(x) = 2$, implying
			$(1+\sqrt{-3})$ is also irreducible. 
			so  $4$ has two factorizations in $Z[\sqrt{-3}]$.
		\end{proof}
	\item[26] In $Z[\sqrt2]$ show that any element of the form
		$(3+2\sqrt2)^n$ is a unit.\\
		Note that unity in $Z[\sqrt2]$ is $1$.
		$$(3+2\sqrt2)(3-2\sqrt2) = 1$$
		$$(3+2\sqrt2)^n (3-2\sqrt2)^n = 1^n = 1 $$
		That is to say that $ (3-2\sqrt2)^n$ is the additive inverse of
	$(3+2\sqrt2)^n$
	\item[32]
		Determine the units in $Z[i]$. \\
		If $x$ is a unit then we know that $N(x) = a^2+b^2 = 1$ which
		clearly only has the solutions, $a = \pm1, b = 0$ and $a = 0 b =
		\pm 1$.
		So $\pm i$ and $\pm 1$ are the only units of $Z[i]$.
	\item[37]
		Show that an integral domain $R$ satisfies the ascending chain
		condition iff every ideal of $R$ is finitely generated. \\
		Let $I$ be a ideal generated by $n$ elements if $I \subset I'$
		then there must be an $x \not\in I', x\in I$. So $I'$ must be generated by
		$n+1$ element so if $(I_1 \subset I_2 \subset I_3 ...) \subset R$ by
		induction (and because $I_1$ is generated by at least one
		element) for
		any arbitrary $k$, $I_k$ contains at least $k$ elements.\\
		Now suppose there is an ideal $I \subseteq R$ that can not be generated by finite
		number of elements, $I = <i_1, i_2, i_3 ... >$. Let $I_n = <i_1,
		i_2, ... i_n>$ we can see that $(I_1 \subset I_2 \subset I_3
		...)
		\subset R$ is an infinite chain of ideals.
	\item[40] Find the inverse of $(1 + \sqrt2)$ in $Z[\sqrt2]$, what is the
		multiplicative order of $1 + \sqrt2$.
		$(1+ \sqrt2)(-1 - \sqrt 2) = 1$,  however $(1+\sqrt2)^n > 1
		\forall n$  so it has an infinite multiplicative order.
	\section{GAP}
	\item[18.1]
		$2$ reducible.
		$3$ irreducible.
		$5$ reducible.
		$7$ irreducible.
		$11$ irreducible.
		$13$ reducible.
		$17$ reducible.
		$19$ irreducible.
		$23$ irreducible.
		$29$ reducible.
		$31$ irreducible.
		$37$ reducible.
		$41$ reducible.
		$43$ irreducible.
		$47$ irreducible.
		$53$ reducible.
		$59$ irreducible.
	\item[18.2]
		$2 \mod4 = 2$,
		$3 \mod4 = 3$,
		$5 \mod4 = 1$,
		$7 \mod4 = 3$,
		$11 \mod4 = 3$,
		$13 \mod4 = 1$,
		$17 \mod4 = 1$,
		$17 \mod4 = 1$,
		$19 \mod4 = 3$,
		$23 \mod4 = 3$,
		$29 \mod4 = 1$,
		$31 \mod4 = 3$,
		$37 \mod4 = 1$,
		$41 \mod4 = 1$,
		$43 \mod4 = 3$,
		$47 \mod4 = 3$,
		$53 \mod4 = 1$,
		$59 \mod4 = 3$ 
	\item[18.3]
		A prime $p\in \Z$ is reducible in $\Z[i]$ iff $p \mod 4 = 1$.
	\item[18.4]
		$2 = 1^2 + 1^2$
		$5 = 1^2 + 2^2$
		$13 = 3^2 + 2^2$
		$17 = 4^2 + 1^2$
		$17 = 5^2 + 2^2$
		$29 = 5^2 + 2^2$
		$37 = 6^2 + 1^2$
		$41 = 5^2 + 4^2$
		$53 = 7^2 + 2^2$\\
		All are irreducible in $Z[i]$.
\end{description}
\end{document}
