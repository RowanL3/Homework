\documentclass[11pt]{article}

\usepackage{scrextend}
\usepackage[a4paper, margin = 1in,footskip =0.25in]{geometry}
\usepackage{enumitem}
\usepackage{amsmath}
\usepackage{amssymb}
\usepackage{amsthm}
\usepackage{mathtools}
\usepackage{amsmath}
\usepackage{graphicx}

\newcommand{\C}{\mathbb{C}}
\newcommand{\Q}{\mathbb{Q}}
\newcommand{\R}{\mathbb{R}}
\newcommand{\Z}{\mathbb{Z}}

\graphicspath{ {images/} }

\DeclarePairedDelimiter\ceil{\lceil}{\rceil}
\DeclarePairedDelimiter\floor{\lfloor}{\rfloor}


\begin{document}
\begin{flushleft}
	Rowan Lochrin \\
	MATH415B - Klaus Lux \\
	1/29/18 \\
	Homework 2
\end{flushleft}
\begin{description}
	\item[10] If $A$,$B$ are ideals in a ring $R$ show that $A+B = \{a+b|a\in A b \in
	B \}$ is an ideal in $R$.\\
	Consider $x \in (A+B)$, for any $r \in R$.
		$$ rx = r(a+b) = ra + rb \text{ for some } a \in A, b \in B$$
		Because $A,B$ are ideals $ra \in A$ and $rb \in B$, meaning $rx
		\in (A+B)$,  $(A+B)$ is an ideal in $R$.

	\item[12] If $A$,$B$ are ideals in a ring $R$ show that $AB = \{a_1b_1 +
		a_2b_2 +...+a_nb_n |a_i\in A b_i \in B\}$ is an ideal.\\
		First note that $a_1b_1 \in A$ (As $B \subseteq R$) so for any
		$x \in AB$, $r \in R$
		\begin{align*}
			rx & = r(a_1b_1 + a_2b_2 +...+a_nb_n) \\
			   & = ra_1b_1 + ra_2b_2 +...+ra_nb_n \\
			   & = a_1' + a_2' ... + a_n' \text { for some }
			   a_1,a_2...,a_n \in A
		\end{align*}
		And from question 10 we 
		know that $a_1 + a_2 \in (A+A) = A$ so by induction:
			$$ rx = a_1' + a_2'+ ... + a_n' \in A$$
	\item[23] Verify that $R/I$ from example $12$ has $16$ elements.\\
		Note that for any matrix $a \in R$.
		$$ a = \begin{bmatrix} a_1 & a_2 \\ a_3 & a_4 \end{bmatrix} =
		 \begin{bmatrix} 2q_1 & 2q_2 \\ 2q_3 & 2q_4 \end{bmatrix} +
		 \begin{bmatrix} r_1 & r_2 \\ r_3 & r_4 \end{bmatrix} $$
			 For some $q_i,r_i $ such that $a_i = 2q_i + r_i, 0 \leq r_i < 2$
			 meaning $$ a + I =  
		 \begin{bmatrix} r_1 & r_2 \\ r_3 & r_4 \end{bmatrix} + I $$
			 so because there are $2$ possible integer choices for
			 each $r$ value there are $2^4 = 16$ distinct cosets in
			 the factor group $R/I$.
	\item[29] In $Z[x]$ the ring of polynomials with integer coefficients, let $ I = \{f(x) \in
		Z[x]|f(0) = 0 \} $. Prove that $ I = <x> $.
	\begin{proof}
		We can see that for any polynomial $p(x) \in Z[x]$, $ p(x)  x $
		evaluated at $0$ is $p(0)  0 = 0$ so $ <x> \subseteq I $. We
		now seek to prove the other direction.\\
		If $p(x) \in I$, then $p(0) = 0$, so we know that there are no
		constant terms in the polynomial meaning that either $p(0) = 0$
		or for $a_1,a_2 ..., a_n \in \Z$ (for some postive $n$)
		$$ p(x) = a_nx^n+a_{n-1}x^{n-1} +... +a_1x = (a_nx{n-1} +
		a_{n-1}x^{n-2} + ... + a_1)x $$
		and $(a_nx{n-1} + a_{n-1}x^{n-2} + ... + a_1) \in Z[x]$ so $p(x)
		\in <x>$. Meaning $I \subseteq <x>$.
	\end{proof}		
	\item[34] In $Z[x]$ the ring of integers with polynomial 
		with integer coefficients, let $ I = \{f(x) \in
		Z[x]|f(0) = 0 \} $. Prove that $ I $ is not a maximal ideal.

	\begin{proof}
		Consider the ideal $ I'  = \{f(x) \in Z[x]|f(n) = 0\} $. For at
		least one $n \in \Z$.
		We know that this is an ideal because for $f(x) \in I', g(x) \in
		R$
			$$ f(n)g(n) = 0g(n) = 0 $$
		So the polynomial $ f(x)g(x) $ has a $0$ at $n$ and is also in
		$I'$
		 
		Note that $I'$ contains all functions with at least one zero
		 so $I' \neq R$ and $I' \neq I$ but $I \subset I'$.
	\end{proof}
\item[37] In $Z[x]$, the ring of polynomials with integer coefficients, let $I =
	< x, 2 >$ be an ideal.
	\begin{enumerate}
	\item Prove $<x, 2> = \{f(x) \in Z[x] | f(0) \text{ is even}\}$
		\begin{proof}
			If $f(x) \in < x ,2> $, $f(x)$ has the form
		$$ f(x) = g(x)x + 2h(x)$$
		For some $g(x),h(x) \in Z[x]$, so $f(0) = 2h(0)$, and
		because $h(0) \in Z$ we know that $f(0)$ is even so
		$f(x) \in I, <x,2> \subseteq I$\\
		If $f(x) \in I$
		\begin{align*}
			f(x) &= a_1x + a_2x^2 + ... a_nx^n + 2b\\
			&= (a_1 + a_2x^1 + ... a_nx^{n-1})x+ 2b
		\end{align*}
		So $f(x) \in <x,2> , I \subseteq <x,2>$
		\end{proof}
	\item Is $I$ prime? Yes. If $ f(x) = g(x)h(x) $, for $f(x) \in I$, $g(x)
		, h(x) \in Z[x]$
			$$f(0) | 2 \Rightarrow g(0) |2 \text{ or } h(0) | 2$$
		so either $h(x) \in I$ or $g(x) \in I$.
	\item Is $I$ maximal? 
		Yes assume there is an ideal $I'$ such that $I \subset I', I \neq I'$,
			so there is some $f(x) \notin I, f(x) \in I'$. From (1)
			we know
		$f(x)$ is not a polynomial that evaluates to an even integer at
			$0$ so $f(x) + 1$ does. Implying $(f(x) + 1) \in I$ (and
		therefore in $I'$) so
			$$ (f(x) + 1) - f(x) = 1 \in  I' $$
			meaning for all $ h(x) \in Z[x] $
			$$ 1h(x) = h(x) \in I' $$
			Implying $I' = Z[x]$
		%$$ f(x) = a_nx^n + a_{n-1}x^{n-1} +... + a_1x + 2k + 1 $$
		%for some $a_n, ... ,a_1,k \in Z$.
	\end{enumerate}
\item[41] For any ideal $I$ of $Z$,let $d$ be the smallest integer that divides
	every element of $I$, we can see that $I \subseteq <d>$ (by our
	definition of division). For any $x \in <d>$ 
	then there exists some $q \in I$ such that $x = dq$.
	So by the definition of an ideal $x \in I$.

\item[54] Let $R$ be a commutative ring without unity, and let $a \in R$.
	Describe the smallest ideal $I$ of $R$ that contains $a$.
	Because $\forall x \in R, xa \in I $, 
	$$ <a> \subseteq I $$
	Now again by the properties of ideals we must include every element that
	is the product of an element in $<a>$ and an element in $R$. So
	$$<<a>> \subseteq I$$
	we must continue in this way until no new elements must be considered
	when we take the span of the elements in the previous set (e.g. If
	$<<a>> = <<<a>>>$) once this is the case the set we have is an ideal.
	Because we have never added elements that are not explicitly required by
	the properties of an ideal we know that this ideal is minimal.
	 We can see from this construction of the minimal ideal that every element of
	 the minimal ideal is a product of $a$.
	a product of $a$.
\item[63] Let $R$ be a commutative ring with unity and let $a,b \in R$. Show
	that $<a,b>$ the smallest ideal of $R$ containing $a$ and $b$.
	Because $R$ has unity
	$$ a = 1_Ra + 0b \in I$$
	$$ b = 0b + 1_Rb \in I$$
	If an ideal $I$ contains $a, b$ for any $r,s$ in $R$
		$$ra, (-1)bs \in I$$
	By the ideal test we know that
	$$ ra - (-1)sb \in I \Rightarrow ra+sb \in I$$


\end{description}



\end{document}

