\documentclass[11pt]{article}

\usepackage{scrextend}
\usepackage[a4paper, margin = 1in,footskip =0.25in]{geometry}
\usepackage{enumitem}
\usepackage{amsmath}
\usepackage{amssymb}
\usepackage{amsthm}
\usepackage{mathtools}
\usepackage{amsmath}
\usepackage{graphicx}

\newcommand{\C}{\mathbb{C}}
\newcommand{\Q}{\mathbb{Q}}
\newcommand{\R}{\mathbb{R}}
\newcommand{\Z}{\mathbb{Z}}

\graphicspath{ {images/} }

\DeclarePairedDelimiter\ceil{\lceil}{\rceil}
\DeclarePairedDelimiter\floor{\lfloor}{\rfloor}


\begin{document}
\begin{flushleft}
	Rowan Lochrin \\
	MATH415B - Klaus Lux \\
	1/20/18 \\
	Homework 1
\end{flushleft}
\begin{description}
	\item[11]
Let $d \in \Z$, prove that $Z[\sqrt{d}] = \{a + b\sqrt{d}: a,b \in \Z \} $
is an integral domain.

\begin{proof}
	We will show first that $Z[\sqrt{d}]$ is a subring of $\R$ by the two step subring test.
	Assume $$(x_1 + y_1\sqrt{d}), (x_2 + y_2\sqrt{d}) \in Z[\sqrt{d}]$$
	Then 
	$$ (x_1 + y_1\sqrt{d}) - (x_2 + y_2\sqrt{d}) =
	 (x_1 - x_2) + (y_1 - y_2)\sqrt{d} \in Z[\sqrt{d}]$$
	Because $(x_1 + x_2), (y_1 + y_2) \in \Z$. Also,
	\begin{align*}
		(x_1 + y_1\sqrt{d})(x_2 + y_2\sqrt{d}) &=
		(x_1x_2) + (x_1y_2\sqrt{d}) + (x_2y_2\sqrt{d}) + (y_1 y_2d) \\
		&= (x_1x_2 + y_1y_2d) + (x_1y_2 + x_2y_1 )\sqrt{d}\in Z[\sqrt{d}]\\
	\end{align*}
	$Z[\sqrt{d}]$ is a subring of $\R$.
	Because there are no zero divisors in $\R$,$\R$ is commutative and $ 1 \in
	Z[\sqrt{d}]$.
	$Z[\sqrt{d}]$ is an integral domain
\end{proof}
		
\item[30] 
Let $d>0 \in \Z$, prove that $Q[\sqrt{d}] = \{a + b\sqrt{d}: a,b \in Q \} $
is a field
\begin{proof}
	By the same argument in question $11$ we know that $ Q[\sqrt{d}] $ is a
	subring of $\R$. So $ Q[\sqrt{d}] $ is a commutative ring.
	To verify that $Q[\sqrt{d}]$ is a field we will also have to show
	every nonzero element has an identity so for any nonzero element $x$:
		$$ x = (a + b\sqrt{d}) \in Q[\sqrt{d}] $$
	We seek to find some $a',b' \in Q$
		$$ x^{-1} = (a' + b'\sqrt{d}) \in Q[\sqrt{d}]$$
	Such that 
	\begin{align*}
		xx^{-1} & = (a + b\sqrt{d})(a' + b'\sqrt{d})\\
		& = aa' + ab' \sqrt{d} + a'b \sqrt{d} + bb'd \\
		& = aa' +  bb'd + (ab' + a'b)\sqrt{d} \\
		& = 1 
	\end{align*}
	Meaning
	\begin{equation}
		aa' +  bb'd + (ab' + a'b)\sqrt{d} = 1 
	\end{equation}
	So 
		$$ ab' + a'b = 0 $$ 
	And
		$$ aa' + bb' = 1 $$
		
	\begin{description}
	\item{If $b = 0$} consider
		$$ a' =  (bd)^{-1} \text{ and } b' = 0 $$
	We know $ (bd)^{-1} $ exists because $ bd \in \Q $, a field.  Note that
			$ bd = 0 $ implies either $ d $ is not positive or  $x =
			0$. So our values for $a'$ and $b'$ solve 1.
	\item{If $b \neq 0$} then 
		$$ a' = \frac{a^2}{a^2 - db^2}, b' = \frac{b^2}{b^2 -
		db^2}$$
		Solve 1.
	\end{description}
	
\end{proof}


\item[41] If $a$ is an idempotent in a commutative ring, show that $1 -  a$ is also
an idempotent.
$$(1-a)^2 = (1-a) - a(1-a) = (1-a) - a + a^2$$
Because $a = a^2$
$$(1-a)^2 = (1-a)$$

\item[42] Construct a multiplication table for $*_\Z[i]$.
	\begin{center}
	\begin{tabular}{ c|c c c c  } 
		$*_{\Z_2[i]}$ & 0  & 1 & i & 1 + i   \\ 
	 \hline
	 0 	& 0 	& 0 & 0 & 0 \\ 
	 1 	& 0 	& 1 & i & i + 1 \\ 
	 i 	& 0 	& i & 1 & i + 1 \\ 
	 1 + i & 0 	& 1 + i & 1 + i  & 0 \\ 
	\end{tabular}
	\end{center}
	$ \Z_2 $ is not a field because $ 1 + i $ has no inverse. It's not a
		integral domain because $ (1+i)^2 = 0 $.

\item[43] The nonzero elements of $\Z_3[i]$ form an Abelian group of
	order 8 under multiplication. Is it isomorphic to $\Z_8, \Z_4
	\oplus  \Z_4 , or \Z_2 \oplus \Z_2 \oplus  \Z_2?$\\
	$ \Z_8 $ we know this because from the theory Abelian group it
	must be isomorphic to one of the there and because
	$$ \{(i+1)^1, ..., (i+1)^8\} = \{1 + i, 2i, 1 + 2i, 2, 2 +
	2i, i, 2 + i, 1\} = \Z_3[i] $$
	So $\Z_3[i]$ is generated by $(i+1)$, implying that it's
	isomorphic to the cyclic group of order 8, $\Z_8$.
	
\item[58] Find the characteristic of $\Z_4 \oplus 4\Z$
	$$\text{ char } \Z_4 \oplus 4\Z = 0$$
	We know this because all non-zero elements of $4\Z$ have infinite order
	under addition.
\item[62] Let $F$ be a finite field with n elements. Prove that $x^{ n - 1 } = 1$ for all
	nonzero $x$ in $F$.
 	\begin{proof}
		Consider the group $F_*$ to be the group of elements in $F$
		under the operation of multiplication. 
		Since $F_*$ is finite by Lagrange's Theorem 
		$$ k  \text{ ord } x = n$$
		For some integer $k$. So
		$$ x^n = x \rightarrow  x^{n-1} = 1 $$
		Meaning $ x^{n-1} = 1 $ in $F$.
 	\end{proof}


\end{description}



\end{document}

